% Compile using LuaLaTeX!

\documentclass{beamer}
%\documentclass[handout]{beamer}

% Beamer

%% Optionen
% %\usetheme{Darmstadt} %same as Frankfurt?
% %\usetheme{Frankfurt} % GTI style, bah
% %\usetheme{PaloAlto} %hässlich
% \usetheme{Warsaw} % abgerundet hübsch
% %\usetheme{Luebeck} % wie warsaw, nur eckiger
% 
% \definecolor{beamer@blendedblue}{rgb}{0.5,0.5,0.3} % changed this
% 
% \setbeamercolor{normal text}{fg=black,bg=white}
% \setbeamercolor{alerted text}{fg=red}
% \setbeamercolor{example text}{fg=green!50!black}
% 
% \setbeamercolor{structure}{fg=beamer@blendedblue}
% 
% \setbeamercolor{background canvas}{parent=normal text}
% \setbeamercolor{background}{parent=background canvas}
% 
% \setbeamercolor{palette primary}{fg=yellow,bg=yellow} % changed this
% \setbeamercolor{palette secondary}{use=structure,fg=structure.fg!100!green} % changed this
% \setbeamercolor{palette tertiary}{use=structure,fg=structure.fg!100!green} % changed this


%Einbinden des Themes
%
% TU
% A Beamer theme with the colors of the TU Dortmund.
% derived from github.com/kdungs/beamertheme-vertex
%
%

\usetheme{Szeged}

%\usepackage{fontspec}
 %   \setmainfont[Ligatures=TeX]{Tex Gyre Adventor}
\usepackage{xcolor}


% Colors
\definecolor{TUgreen}{HTML}{649600}
\definecolor{TUgreenLighter}{HTML}{A5CB5B}
\definecolor{TUgreenDarker}{HTML}{54711C}
\definecolor{TUalert}{HTML}{A10000}
\definecolor{TUalertLighter}{HTML}{D03434}
\definecolor{TUexample}{HTML}{006496}
\definecolor{TUexampleLighter}{HTML}{8735B5}
\definecolor{PeP}{HTML}{565656}
\definecolor{ltgrey}{HTML}{CCCCCC}
\definecolor{ownDarkOr}{HTML}{0101DF}
\definecolor{ownDarkBlue}{HTML}{0080FF}
\definecolor{lightGreen}{HTML}{00FF00}

% Titlepage
%  Title
\setbeamerfont{title}{size=\Huge}
\setbeamercolor{title}{fg=white, bg=TUgreen}
%  Subtitle
\setbeamerfont{subtitle}{size=\large}
%  Author
%\setbeamerfont{author}{size=\large}
%  Institute
\setbeamerfont{institute}{series=\bfseries, size=\small}
\setbeamercolor{institute}{fg=PeP}

% Frametitle
\setbeamerfont{frametitle}{size=\Huge}
\setbeamercolor{frametitle}{fg=white, bg=TUgreen}

% Framesubtitle
\setbeamerfont{framesubtitle}{size=\large}
\setbeamercolor{framesubtitle}{fg=white, bg=TUgreenDarker}

% Structure
\setbeamerfont{structure}{size=\Large}
\setbeamercolor{structure}{fg=TUgreen}

% Blocks
\setbeamertemplate{blocks}[square]
\setbeamerfont{block title}{}
\setbeamercolor{block title}{fg=TUgreen, bg=white}
\setbeamercolor{block title alerted}{use=alerted text,fg=TUalert, bg=white}
\setbeamercolor{block title example}{use=example text,fg=TUexample, bg=white}

% Items
\setbeamertemplate{items}[circle]

% Other
\setbeamertemplate{navigation symbols}{} %no nav symbols

% Customizations


\setbeamertemplate{frametitle}{
	\ifx\insertframesubtitle\empty
		*\vspace{-0.5em}
		\begin{beamercolorbox}[wd=0.92\paperwidth, ht=0.9em, dp=0.2em, leftskip=0.04em, left]{frametitle}
			\hspace{.25em}\usebeamercolor{frametitle}\usebeamerfont{frametitle}\insertframetitle 
		\end{beamercolorbox}
  	\else
  		*\vspace{-0.5em}
    	\begin{beamercolorbox}[wd=0.92\paperwidth, ht=0.9em, dp=0.2em, leftskip=0.04em, left]{frametitle}
    		\hspace{.25em}\usebeamercolor{frametitle}\usebeamerfont{frametitle}\insertframetitle 
    	\end{beamercolorbox}
    	\begin{beamercolorbox}[wd=0.92\paperwidth, ht=0.7em, dp=0.2em, leftskip=0.04em, left]{framesubtitle}
    		\hspace{.25em}\usebeamercolor{framesubtitle}\usebeamerfont{framesubtitle}\insertframesubtitle
    	\end{beamercolorbox}
  	\fi
}

\defbeamertemplate*{footline}{TU}{
    \leavevmode
    \begin{beamercolorbox}[wd=0\paperwidth, ht=0.1em, dp=1em, left, leftskip=1em]{}
        \textbf{\insertshorttitle}
    \end{beamercolorbox}
    \begin{beamercolorbox}[wd=.8\paperwidth, ht=0.1em, dp=1em, center, rightskip=4em]{}
        \insertshortauthor\quad(\insertshortinstitute)
    \end{beamercolorbox}
    \begin{beamercolorbox}[wd=.1\paperwidth, ht=0.1em, dp=1em, right, rightskip=-4em]{}
        \insertframenumber{} / \inserttotalframenumber
    \end{beamercolorbox}
}


% Bild beim Footer
%\pgfdeclareimage[height=0.4cm]{university-logo}{images/tud_logo_cmyk.pdf}
%\logo{\pgfuseimage{university-logo}}


% more flexible than babel. Requires LuaLaTeX or XeTeX
\usepackage{polyglossia}
\setdefaultlanguage{german}

% PDFLatex? -> fontenc and inputenc
% T1 for western-europe coding
%\usepackage[utf8]{inputenc}
%\usepackage[T1]{fontenc}
%\usepackage{lmodern}

% LuaTeX and XeTeX (but LuaLaTeX is cooler) -> fontspec
\usepackage{fontspec}

% extra symbols
\usepackage{amsmath}
\usepackage{amsfonts}
\usepackage{amssymb}
\usepackage{amsthm}


\usepackage{multimedia}

\usepackage{hyperref}
% Hyperref Optionen - makeatletter/AtBeginDocument erlaubt uns Titel/Author dynamisch zu setzen, wenn gewollt
\makeatletter
	\hypersetup{
	unicode = true,
	pdftoolbar = true,
	pdfmenubar = true,
	pdffitwindow = true,
	pdftitle = {\@title},
	pdfauthor = {\@author},
	pdfsubject = {FOSS-AG},
	colorlinks = true,
	linkcolor = black,
	citecolor = black,
	filecolor = black,
	urlcolor = black
	}
\makeatother



% TOC bei jeder Subsection
% \AtBeginSubsection[]
% {
%   \begin{frame}<beamer>{Outline}
%     \tableofcontents[currentsection,currentsubsection]
%   \end{frame}
% }

%% Fußzeile mit Framenumber
%\setbeamertemplate{footline}{
%	\begin{beamercolorbox}[wd=0.5\textwidth,ht=3ex,dp=1.5ex,leftskip=.5em,rightskip=.5em]{author in head/foot}
%		\usebeamerfont{author in head/foot}
%		\insertframenumber/\inserttotalframenumber\hfill%\insertshortauthor
%	\end{beamercolorbox}
%	\vspace*{-4.5ex}\hspace*{0.5\textwidth}
%	\begin{beamercolorbox}[wd=0.5\textwidth,ht=3ex,dp=1.5ex,left,leftskip=.5em]{title in head/foot}
%		\usebeamerfont{title in head/foot}
%		\insertshorttitle
%	\end{beamercolorbox}
%}

\usepackage[]{graphicx}
\usepackage{listings}
%\lstset{tabsize=2}
% or
%\lstset{
%	numbers = left,
%	frame = tb,
%	mathescape = true,
%	language = Python,
%	escapeinside = {(*@}{@*)},
%	emph = {None, True, False, lock, unlock},
%	emphstyle = {\bfseries},
%	basicstyle = \ttfamily
%}

% Bibliographie
%\usepackage[numbers]{natbib}
%\usepackage{booktabs}
%\usepackage{float}
%\restylefloat{figure}

\usepackage{pdfpages}
\usepackage{color}
\definecolor{brightGray}{gray}{0.9}
\definecolor{darkGray}{gray}{0.2}
\definecolor{white}{rgb}{1,1,1}
\definecolor{black}{rgb}{0,0,0}
\definecolor{red}{rgb}{1,0,0}


% Abkuerzungen richtig formatieren
\usepackage{xspace}
\newcommand{\vgl}{vgl.\xspace} 
\newcommand{\zB}{z.\,B.\xspace}
\newcommand{\dahe}{d.\,h.\nolinebreak[4]\xspace}
\newcommand{\uvm}{u.\,v.\,m.\xspace}
\newcommand{\usw}{u.\,s.\,w.\xspace}
\newcommand{\sg}{s.\,g.\xspace}
\newcommand{\iA}{i.\,A.\xspace}
\newcommand{\sa}{s.\,a.\xspace}
\newcommand{\su}{s.\,u.\xspace}
\newcommand{\ua}{u.\,a.\xspace}
\newcommand{\og}{o.\,g.\xspace}
\newcommand{\oae}{o.\,\"a.\xspace}
\newcommand{\oBdA}{o.\,B.\,d.\,A.\xspace}
\newcommand{\OBdA}{O.\,B.\,d.\,A.\xspace}

% footnote ohne marker blfootnote
\newcommand\blfootnote[1]{%
  \begingroup
  \renewcommand\thefootnote{}\footnote{#1}%
  \addtocounter{footnote}{-1}%
  \endgroup
}


%\author{
%  Gajda, Michael \\
%  \texttt{michael.gajda@tu-dortmund.de}
%  \and \\
%  Schneider, Josef \\
%  \texttt{josef.schneider@tu-dortmund.de}
%}
\author{FOSS-AG}
\date{\today}

\subject{Linux Days Dortmund}
\title{Das Linux Dateisystem}
\subtitle{Linux Days Dortmund \textcolor{black}{2017}}
%\institute[TU-Dortmund]{Fakultät für Informatik, Technische Universität Dortmund}
%%\institute[TU-Dortmund]{Department of computer science, Technische Universität Dortmund}

%\usebackgroundtemplate{\includegraphics[width=\paperwidth,height=\paperheight]{Latex_Template/Background.png}}

\begin{document}

% Titlepage - Hack um Sidebar ohne Inhalt anzuzeigen
\setbeamertemplate{sidebar left}{}
\begin{frame}
    \titlepage
\end{frame}

\begin{frame}{Frametext1}
Foo Title
\end{frame}

\begin{frame}{Frametext1}
Bar Title
\end{frame}

% Sidebarinhalt wieder mit TOC aber ohne author
\setbeamertemplate{sidebar left}[sidebar theme]

\makeatother
%\section{FHS}
%\subsection{nop}
\begin{frame}{Frametext1}
FHS
\end{frame}

\section[/root]{/root, als Nutzer}
\subsection{Subsectiontext}
\begin{frame}{Frametext1}
"home" Ordner
\end{frame}


%\section[/boot]{/boot, Engine start!}
\subsection{Startvorgang}
\begin{frame}{Ein System wird gestartet}
\begin{itemize}
 \item BIOS/EFI
 \item GRUB
 \item Kernel
 \item Der Rest
 \item LogIn
\end{itemize}

\end{frame}

\subsection{Inhalt}
\begin{frame}{Inhalt von /boot}
Je nach Distro und verfügbarer Größe:
\begin{itemize}
 \item Kernel (mehrfach)
 \item initrm
 \item grub
\end{itemize}
\begin{center}
{\large Vorsicht!}
 
\end{center}

\end{frame}

\section{Übersicht}
\begin{frame}{Überblick}

\begin{minipage}{0.48\linewidth}
\begin{itemize}
\item /bin \onslide<5->{$ \leftarrow $}
\item /boot \onslide<5->{$ \leftarrow $}
\item /dev \onslide<3->{$ \leftarrow $}
\item /etc \onslide<3->{$ \leftarrow $}
\item /home \onslide<2->{$ \leftarrow $}
\item /lib \onslide<5->{$ \leftarrow $}
\item /media \onslide<5->{$ \leftarrow $}
\item /mnt \onslide<4->{$ \leftarrow $}
\item /opt \onslide<4->{$ \leftarrow $}
\end{itemize}
\end{minipage}%
\hspace{0.02cm}
\begin{minipage}{0.48\linewidth}
\begin{itemize}
\item /proc \onslide<5->{$ \leftarrow $}
\item /root \onslide<4->{$ \leftarrow $}
\item /run \onslide<3->{$ \leftarrow $}
\item /sbin \onslide<5->{$ \leftarrow $}
\item /srv \onslide<4->{$ \leftarrow $}
\item /sys \onslide<5->{$ \leftarrow $}
\item /tmp \onslide<4->{$ \leftarrow $}
\item /usr \onslide<5->{$ \leftarrow $}
\item /var \onslide<4->{$ \leftarrow $}
\end{itemize}
\end{minipage}

\vspace{1cm}
{\small
\onslide<2>{Gelegenheits-Nutzer}
\onslide<3>{ Normal-Nutzer}
\onslide<4>{ Fortgeschritten}
\onslide<5>{ Nerd ;-)}
}
\end{frame}


\section[Dateisysteme]{Dateisysteme}
\begin{frame}{EXT4, Btrfs, RaiserFS, ZFS, FAT, NTFS, NFS…}
\begin{itemize}
\item Linux beherrscht extrem viele Dateisysteme
\item Ein Dateisystem kann auch 'virtuell' sein
\item Netzwerkzugriff per Netzwerk-Dateisystem (NFS, Samba…)
\item Beliebig in den Baum einhängbar
\end{itemize}
\end{frame}

\section[Dist.Eigen.]{Distro Eigenheiten}
\begin{frame}{Dienst-Daten}
\begin{itemize}
\item Kleinere Variationen je nach Distribution
\item Beispiel: Serverdaten /srv vs. /var
\end{itemize}
\end{frame}

\section[/var/log]{/var/log, Logging}
\begin{frame}{Logging}
\begin{itemize}
\item /var/log
\item Speicherort für Log-Dateien
\end{itemize}
\end{frame}

\section[alles Datei]{Philosophie}
\begin{frame}{Gleichmacherei}
Alles in Dateien:
\begin{itemize}
 \item Festplatten
 \item Prozesse
 \item Kameras
 \item Katzenbilder
\end{itemize}

Vorteile:
\begin{itemize}
 \item Einfache Schnittstelle
 \item Immer gleicher Zugriff 
\end{itemize}

\end{frame}



%\section[/proc, /sys]{/proc, /sys}
\begin{frame}{Systemparameter}
\begin{itemize}
\item Virtuelles Dateisystem
\item Kernel-Parameter
\item Treiber-Einstellungen
\item Prozesse
\item Subsystem-Zustandsinformationen
\end{itemize}
\end{frame}

\begin{frame}{/dev, Die DEVices.}
Beispiele:\\
/dev
\begin{itemize}
 \item [/] sda (1,2,..)
 \item [/] sdb (1,2,..)
 \item [/] tty (1,2,..)
 \item [/] cpu
 \item [/] std (err,in,out)
 \item [/] zero
\end{itemize}
(alles eine Dateien…)
\end{frame}



%\section{x}
\subsection{x}


\section[/run]{/run, alles läuft}
\begin{frame}{/run, alles läuft}
Mit /run lebt das System:
\begin{itemize}
 \item Lauzeitdaten
 \item Mount für Sticks, Platten
\end{itemize}


 
\end{frame}



%? %\include{inc/inc_overview2}

\section[/home]{/home, sweet Home}
\begin{frame}{Wer braucht schon /home?}
/home ist für:
\begin{itemize}
 \item dich und
 \item mich und
 \item jeden anderen (menschlichen Benutzer)
\end{itemize}
\end{frame}

\begin{frame}{Wo bin ich Hause?}
/home
\begin{itemize}
  \item [/] michael
  \item [/] josef
\end{itemize}
\vfill
Ort für \note{ja, gerne groß geschrieben..} {\small (echt!)}
\begin{itemize}
 \item [/] Dokumente
 \item [/] Downloads
 \item [/] Bilder
\end{itemize}
\vfill
($\longrightarrow$ änlich zu \textbackslash Users)

\vfill
\end{frame}

\section{x}
\subsection{x}


\section[/etc, \textasciitilde/.*]{/etc, \textasciitilde/.*, ~/.config/*}
\begin{frame}{Ein Konzept für Konfigurationen \tiny (Und es ist keine besch*** Datenbank!)}
\begin{itemize}
\item Unterschiede:
	\begin{itemize}
	\item Systemweit:
		\begin{itemize}
		\item Einstellungen (/etc)
		\item Temporäre Datne (/tmp)
		\item Programmdaten (/var)
		\item 
		\end{itemize}
	\item Benutzerspezifisch:
	\begin{itemize}
		\item Einstellungen (\textasciitilde/.config/…)
		\item Temporäre Dateien (\textasciitilde/.cache/…)
		\item Programmdaten (\textasciitilde/.local/…)	
	\end{itemize}
	\end{itemize}
\item In seltenen Fällen Ausnahmen
\end{itemize}
\end{frame}

\begin{frame}{Der Rest ist schnell erklärt}
\begin{itemize}
\item Was man noch kennen sollte: \\
\begin{itemize}
\item[/mnt] Kurzzeitiges Einhängen anderer, benötigter Dateisysteme
\item[/media] Datenträger, bei vielen Distros /run/media
\end{itemize}
\end{itemize}
\end{frame}

\section[Permissions]{Permissions}
\begin{frame}{Wer darf was?}
\texttt{\small -rw-r--r-- 1 draget draget 16  3. Dez 10:12 yay.txt}
\begin{itemize}
\item Rechte
\item Benutzer
\item Gruppe
\item Größe
\end{itemize}

\vspace{1cm}
Wie sind die Rechte zu lesen?
\end{frame}

\begin{frame}{chmod 777! \tiny(Bitte nicht machen)}
\texttt{\huge
\textcolor{red}{-}\textcolor{blue}{rw-}\textcolor{green}{r--}\textcolor{orange}{r--} …
}

\begin{minipage}{0.48\linewidth}
\vspace{0.5cm}
\begin{itemize}
\item \textcolor{red}{Typ}
\begin{itemize}
\item[-] Reguläre Datei
\item[c] Character-Gerät
\item[b] Block-Gerät
\item[d] Verzeichnis
\end{itemize}
\item \textcolor{blue}{Besitzer-Rechte}
\begin{itemize}
\item[r] Lesen
\item[w] Schreiben
\item[x] Ausführen / Nutzen
\item[sSG] Speziell
\end{itemize}
\end{itemize}
\end{minipage}%
\begin{minipage}{0.48\linewidth}
\vspace{0.5cm}
\begin{itemize}
\item \textcolor{green}{Grp.-Mitglieder-Rechte}
	\begin{itemize}
	\item[…] -analog-
	\end{itemize}
\item \textcolor{orange}{Jeder-Rechte}
	\begin{itemize}
	\item[…] -analog-
	\end{itemize}
\item \texttt{chmod xxx <Datei>}
	\begin{itemize}
	\item Binär zählen! rw- $\rightarrow$ 110 = 6
	\end{itemize}
\item \texttt{chmod u=rw,g-x,o=r <Datei>}
\end{itemize}
\end{minipage}

\end{frame}

\end{document}
